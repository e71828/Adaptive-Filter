\documentclass[11pt]{article}
\usepackage[margin=2cm,a4paper, headheight= 14pt]{geometry}
\usepackage{xeCJK}
%% 也可以选用其它字库:
\setCJKmainfont{Noto Serif CJK SC}[%
    UprightFont    = * Light,
    BoldFont       = * Bold,
    ItalicFont     = AR PL UKai CN,
    BoldItalicFont = AR PL KaitiM GB,
]
\setCJKsansfont{Noto Sans CJK SC}
\setCJKmonofont{Noto Sans Mono CJK SC}
\newcommand{\kaishu}{\CJKfontspec{FandolKai}}


%% math font
\usepackage{amsmath,amsthm,amssymb}
\IfFontExistsTF{XITS-Regular.otf}{%
    \RequirePackage{unicode-math}% unicode font configuration <XITS|STIX2>
    \def\boldsymbol#1{\symbfit{#1}}% treat obsoleteness
    \providecommand{\Vector}[1]{\symbfit{#1}}% general vectors in bold italic
    \providecommand{\unitVector}[1]{\symbfup{#1}}% unit vectors in bold roman
    \providecommand{\Matrix}[1]{\symbfup{#1}}% matrix in bold roman
    \providecommand{\unitMatrix}[1]{\symbfup{#1}}% identity matrix in bold roman
    \providecommand{\Tensor}[1]{\symbfsfup{#1}}% tensor in sans-serif bold italic
    \providecommand{\unitTensor}[1]{\symbfsfup{#1}}% identity tensor in sans-serif bold
    \newcommand*{\tran}{^{\mkern-1.5mu\mathsf{T}}}
    \newcommand*{\conj}[1]{\overline{#1}}
    \newcommand*{\hermconj}{^{\mathsf{H}}}
    \setmainfont[NFSSFamily=entextrm]{XITS}[
      Extension      = .otf,
      UprightFont    = *-Regular,
      BoldFont       = *-Bold,
      ItalicFont     = *-Italic,
      BoldItalicFont = *-BoldItalic,
      Ligatures=TeX,
    ]
    \setsansfont[NFSSFamily=entextsf]{XITS}[
      Extension      = .otf,
      UprightFont    = *-Regular,
      BoldFont       = *-Bold,
      ItalicFont     = *-Italic,
      BoldItalicFont = *-BoldItalic,
      Ligatures=TeX,
    ]
    \setmonofont[NFSSFamily=entexttt]{XITS}[
      Extension      = .otf,
      UprightFont    = *-Regular,
      BoldFont       = *-Bold,
      ItalicFont     = *-Italic,
      BoldItalicFont = *-BoldItalic,
      Ligatures=TeX,
    ]
    \setmathfont{XITSMath-Regular}[
      Extension    = .otf,
      BoldFont     = XITSMath-Bold,
      Ligatures=TeX,
      StylisticSet = 1,
    ]
    \setmathfont{XITSMath-Regular}[
      Extension    = .otf,
      range={cal,bfcal},
      Ligatures=TeX,
      StylisticSet = 1,
    ]
    \setmathfont{XITSMath-Regular}[
      Extension    = .otf,
      range={scr,bfscr},
      Ligatures=TeX,
      StylisticSet = 2,
    ]
    \setmathrm{XITSMath-Regular}[
      Extension    = .otf,
      BoldFont     = XITSMath-Bold,
      Ligatures=TeX,
      StylisticSet = 1,
    ]
    \setmathsf{XITSMath-Regular}[
      Extension    = .otf,
      BoldFont     = XITSMath-Bold,
      Ligatures=TeX,
      StylisticSet = 1,
    ]
    \setmathtt{XITSMath-Regular}[
      Extension    = .otf,
      BoldFont     = XITSMath-Bold,
      Ligatures=TeX,
      StylisticSet = 1,
    ]
}{%
    \RequirePackage{newtxtext}% main font
    \RequirePackage[cmintegrals]{newtxmath}% times font, load after amsmath and newtxtext packages
    \RequirePackage{mathrsfs}% enable \mathscr for script alphabet
    \RequirePackage[cal=cm]{mathalfa}% map styles for calligraphic \mathcal and script \mathscr alphabet
}

\usepackage{enumitem} %% use for itemindent
\newcommand\setItemnumber[1]{\setcounter{enumi}{\numexpr#1-1\relax}}  %% setcounter
%% https://tex.stackexchange.com/questions/255197/how-can-i-change-the-number-of-enumerate-items
\usepackage{ulem}

\usepackage{fancyhdr}
\pagestyle{fancy}
\fancyhf{}
\rhead{Overleaf}
\lhead{Fundamentals of Adaptive Filtering}
\rfoot{Page \thepage}

\usepackage[ruled, vlined]{algorithm2e}

\AtBeginDocument{
	  \DeclareSymbolFont{AMSb}{U}{msb}{m}{n}
	  \DeclareSymbolFontAlphabet{\mathbb}{AMSb}
	  \setmathfontface\mathit{XITS-Italic.otf}
}
\title{在Overleaf平台上使用C\TeX 完成作业:Chapter 2}
\author{唐金峰}
\begin{document}
\maketitle
\section*{Problems}
\begin{enumerate}[itemindent= 0pt]
    \setItemnumber{2}
    \item  Consider two complex random processes:
        \begin{enumerate}[itemindent= -5pt]
            \item \sout{Pseudo-covariance}: 
                $ \sigma_{xy}^2 (k, l)
                = \mathscr{E} \left\{ [x(k)-m_{x}(k)] [y(l)-m_{y}(l)]^* \right\}
                = r_{xy}(k,l) - m_x(k)m_y^*(l)
                $
            \item Since jointly WSS:
                $ \sigma_{xy}^2 (k, l)
                = r_{xy}(k-l) - m_x(k)m_y^*(l)
                $
            \item Since orthogonal:
                $ \sigma_{xy}^2 (k, l) = - m_x(k)m_y^*(l) $; we conclude: if $m_x(k) = 0$ or $m_y(l)=0$ 
        \end{enumerate}
\end{enumerate}

\begin{enumerate}[itemindent= 0pt]
    \setItemnumber{5}
    \item  A second-order adaptive filter:
        \begin{enumerate}[itemindent= -5pt]
            \item The models for the signals involved are described by
                $$ \Vector{x}(k) = [x(k)\; x(k-1)\; x(k-2)]\tran = \alpha_1 \Vector{x_1}(k) + \alpha_2 \Vector{x_2}(k) $$
                $$ x_{i}(k)= -a_{i} x_{i}(k-1)+u(k) $$
                for $i=1,2;\; a_1 = a,\; a_2= -b$; given autocorrelation
                $$
                r_{x_1}(l) = \mathscr{E}\left[x_{i}(k) x_{i}(k-l)\right]=
                \frac{\sigma^{2}}{1-a_{i}^{2}} (-a_{i})^{|l|} = (-a_{i})^{|l|};\;\text{for unit variance: } (\sigma^{2} = 1-a_{i}^{2})
                $$
                $$
                \Matrix{R} = \mathscr{E}\{  \Vector{x_1}(k)  \Vector{x_1}(k)\tran \}
                + \mathscr{E}\{  \Vector{x_2}(k)  \Vector{x_2}(k)\tran \}
                = \alpha_1^2 \begin{bmatrix}
                1 & a & a^2\\
                a & 1 & a  \\
                a^2&a & 1
                \end{bmatrix}
                + \alpha_2^2 \begin{bmatrix}
                1 & -b & b^2\\
                -b & 1 & -b \\
                b^2&-b & 1
                \end{bmatrix}
                $$
                $$\Vector{p} = \mathscr{E} \{  d(k)  \Vector{x}(k) \}
                = \alpha_3 \mathscr{E} \{  x_2(k)  \Vector{x}(k) \}
                = \alpha_2 \alpha_3 \begin{bmatrix}
                1 & -b & b^2
                \end{bmatrix}\tran
                $$
            \item  The Wiener solution can then be expressed as:
            $$
            w_0 = \Matrix{R}^{-1}\Vector{p}
            $$
        \end{enumerate}
\end{enumerate}

\begin{enumerate}[itemindent= 0pt]
    \setItemnumber{8}
    \item  In the prediction case the input signal is $x(k)$ and the desired signal is $x(k+l)$.
    
    Since:
    $$
    \mathscr{E} \{ x(k)x(k-l) \} = \frac{1}{2} \cos(\omega_0 l)
    + \frac{1}{2} \cos(2\omega_0 k - \omega_0l) + \frac{\sigma^2}{1-a^2} (-a)^{|l|}
    $$
    Assume $ n_1(k) $ with unit variance, the input signal correlation matrix is
    $$
    \Matrix{R} = 
    \begin{bmatrix}
        \frac{1}{2} + \frac{1}{2} \cos(2\omega_0 k) +1 &
        \frac{1}{2}\cos(\omega_0) + \frac{1}{2} \cos(2\omega_0 k -\omega_0)-a \\
        \frac{1}{2}\cos(\omega_0) + \frac{1}{2} \cos(2\omega_0 k -\omega_0)-a &
        \frac{1}{2} + \frac{1}{2} \cos(2\omega_0 k) +1
    \end{bmatrix}
    $$
    Vector $\Vector{p}$ is described by
    $$
    \Vector{p} = \begin{bmatrix}
        \mathscr{E} \{ x(k+l)x(k) \} \\
        \mathscr{E} \{ x(k+l)x(k-1) \}
    \end{bmatrix} = \begin{bmatrix}
        \frac{1}{2} \cos(\omega_0 l)
        + \frac{1}{2} \cos(\omega_0 (2k+l)) + (-a)^{|l|} \\
        \frac{1}{2} \cos(\omega_0 (l+1))
        + \frac{1}{2} \cos(\omega_0 (2k+l-1)) + (-a)^{|l+1|}
    \end{bmatrix}
    $$
    The expression for the optimal coefficient vector is easily derived.
    $$
    w_0 = \Matrix{R}^{-1}\Vector{p}
    $$
    where in the above equation the value of $l$ is considered positive. Since part of input signal is deterministic and nonstationary, the autocorrelation is time dependent.
    
    Specially, for $\omega_0 = 0$, the autocorrelation is time indpendent. But there is no need for Wiener filter, we predict with the property of AR(1) process.
\end{enumerate}


%% This is needed if you want to add comments in
%% your algorithm with \Comment
\SetKwComment{Comment}{/* }{ */}
\normalem %%%% disable auto underline
\begin{algorithm}[hbt!]
    \caption{Gradient Descent}\label{alg:gradient}
    \KwData{$\Vector{w}(0) = \mathbf{0}$, $\Vector{p}$, $\Matrix{R}$}
    \KwResult{$ \arg\min \xi:=\{ \Vector{w} \in \mathbb{R}^2: \xi=\sigma_d^2 -2 \Vector{w}\tran \Vector{p} + \Vector{w}\tran \Matrix{R} \Vector{w} \}  $}
    $k \gets 0$\;
    $\Vector{g}_w(0) \gets -2 \Vector{p} + 2\Matrix{R} \Vector{w}(0)$\;
    \While{$\| \Vector{g}_w(k) \| \geq \epsilon $}{
        $\Vector{w}(k+1) = \Vector{w}(k) - \mu \Vector{g}_w(k)$\;
        $k \gets k+1$  \Comment*[r]{Can be reassigned directly}
        $\Vector{g}_w(k) = -2 \Vector{p} + 2\Matrix{R} \Vector{w}(k)$\;
    }
\end{algorithm}

\begin{algorithm}[hbt!]
    \caption{Newton's Method}\label{alg:newton}
    \KwData{$\Vector{w}(0) = \mathbf{0}$, $\Vector{p}$, $\Matrix{R}$}
    \KwResult{$ \arg\min \xi:=\{ \Vector{w} \in \mathbb{R}^2: \xi=\sigma_d^2 -2 \Vector{w}\tran \Vector{p} + \Vector{w}\tran \Matrix{R} \Vector{w} \}  $}
    $k \gets 0$\;
    $\Vector{g}_w(0) \gets -2 \Vector{p} + 2\Matrix{R} \Vector{w}(0)$\;
    \While{$\| \Vector{g}_w(k) \| \geq \epsilon $}{
        $\Vector{w}(k+1) = \Vector{w}(k) -\tfrac{1}{2} \Matrix{R}^{-1}\Vector{g}_w(k)$\;
        $k \gets k+1$\;
        $\Vector{g}_w(k) = -2 \Vector{p} + 2\Matrix{R} \Vector{w}(k)$ \Comment*[r]{Reached in one step}
    }
\end{algorithm}

\end{document}
