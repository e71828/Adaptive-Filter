\documentclass[11pt]{article}
%%begin novalidate
\catcode`_=\active
\newcommand_[1]{\ensuremath{\sb{\scriptscriptstyle #1}}}
%%end novalidate

\usepackage{myStyle}
\title{在Overleaf平台上使用C\TeX 完成作业:Chapter 3}
\author{唐金峰}
\begin{document}
\maketitle
\section*{Problems}
\begin{enumerate}[itemindent= 0pt]
    \setItemnumber{2}
    \item  The singal:
        $$x(k)=-0.85x(k-1)+n(k)$$
        \begin{enumerate}[itemindent= -5pt]
            \item Compute the Wiener solution, according to example 2.1(b):
            $$
            \Vector{w}(0)=
            \begin{bmatrix}
            (-a)^L \\ 0
            \end{bmatrix}
            $$
            Therefore,
            $$
            \hat{x}(k+L)=(-a)^{L} x(k) =(-0.85)^{L} x(k)
            $$
            \item Choose an appropriate value for $\mu$ and plot the convergence path for the LMS algorithm on the MSE error surface.
            $$
            \Matrix{R} = \frac{\sigma_n^2}{1-a^2}
            \begin{bmatrix}
            1 & -a \\
            -a & 1
            \end{bmatrix}
            $$
            we get,
            $$
            \lambda_{max} = \frac{\sigma_n^2}{1-a} = 2
            $$
            So choose $\mu= \dfrac{1/5}{\lambda_{max} \times 2 + \frac{\sigma_n^2}{1-a^2} }=0.032 $, and in a prediction case, also choose $L=1$.
            $$
            \Vector{p} = \frac{\sigma_n^2}{1-a^2}\begin{bmatrix}
            (-a)^L \\ (-a)^{L+1}
            \end{bmatrix}=\frac{\sigma_n^2}{1-a^2}\begin{bmatrix}
            -a \\ a^2
            \end{bmatrix}
            $$
            Therefore,
            $$
            \Vector{w}_0 =\Matrix{R}^{-1}\Vector{p}=\begin{bmatrix}
            -a \\ 0
            \end{bmatrix}=\begin{bmatrix}
            -0.85 \\ 0
            \end{bmatrix}
            $$
            \item Plot the learning curves for the MSE and the filter coefficients in a single run as well as for the average of 25 runs.
        \end{enumerate}
\end{enumerate}


\begin{enumerate}[itemindent= 0pt]
    \setItemnumber{5}
    \item  The updating equation:
        $$\Vector{w}(k+1) = \Vector{w}(k) +2 \tilde{\mu} e(k)\Vector{x}(k) +\gamma [\Vector{w}(k)-\Vector{w}(k-1)] $$
        \begin{enumerate}[itemindent= -5pt]
            \item Compute the range of values of $\mu$:
            \begin{equation}\label{eq: LMF}
                \Vector{b}(k) :=  \gamma \Vector{b}(k-1) + (1-\gamma)\Vector{g}_w(k)
            \end{equation}

            Assume,
            $$
            \begin{aligned}
            \Vector{w}(k+1) &=\Vector{w}(k)-\mu \Vector{b}(k) \\
            \gamma \Vector{w}(k) &=\gamma \Vector{w}(k-1)-\gamma \mu \Vector{b}(k-1) \\
            \Vector{w}(k+1)-\gamma \Vector{w}(k) &=\Vector{w}(k)-\gamma \Vector{w}(k-1)-\mu \Vector{b}(k)+\gamma \mu \Vector{b}(k-1)\\
            \Vector{w}(k+1)-\gamma \Vector{w}(k) &=\Vector{w}(k)-\gamma \Vector{w}(k-1)- (1-\gamma) \mu \Vector{g}_w(k)
            \end{aligned}
            $$
            Lowpass filtering the noisy gradient Rename the noisy gradient, passing the signals $ \Vector{g}_w(k) $ through low pass filters will prevent the large fluctuations of direction during adaptation process.

            Momentum LMS algorithm LPF is an IIR filter of first order.

            In these equations, it can be \emph{proved} that the $\mu$ gets the same value range as the euqation (3.19).
            So,
            $$
            \Vector{w}(k+1) = \Vector{w}(k) +2 \tilde{\mu} e(k)\Vector{x}(k) +\gamma [\Vector{w}(k)-\Vector{w}(k-1)]
            $$
            where,
            $
            \tilde{\mu} = (1-\gamma) \mu.
            $

            According to the euqation (3.19),
            $$
            0<\tilde{\mu}<\frac{1-\gamma}{\lambda_{\max }}
            $$
            \item The objective function this algorithm actually minimizes:

            Since,
            $$
            \Vector{g}_w(l)\;\mathrm{d} \Vector{w} = e^2(l)
            $$
            So, according to Eq~\eqref{eq: LMF}
            \begin{align*}
                \xi(k) &=  (1 - \gamma) \sum_{l=0}^{k}\gamma^{k-l}e^2(l)
            \end{align*}

            \item Show that this algorithm can have faster convergence and higher misadjustment than the LMS algorithm.

            Answer: The convergence rate may decrease.
        \end{enumerate}
\end{enumerate}

\begin{enumerate}[itemindent= 0pt]
    \setItemnumber{12}
    \item  System identification:
    $$
    \begin{aligned}
    x(k)&= -1.2x(k-1)-0.81x(k-2)+n_x(k)\\
    \sigma_x^2 &=1\\
    \textsc{unknown:} H(z)&=1+ 0.9z^{-1}+ 0.1z^{-2}+ 0.2z^{-3}\\
    \sigma_n^2 &=0.04
    \end{aligned}
    $$
    \begin{enumerate}[itemindent= -5pt]
    \item Choose an appropriate $\mu$, run an ensemble of 20 experiments, and plot the average learning curve.
    $$
    r_{* \pi}[k]=\sigma_x^{2} \frac{\frac{1+r^{2}}{1-r^{2}} \sqrt{1+\left(\frac{1-r^{2}}{1+r^{2}}\right)^{2} \cot ^{2} \left(2 \pi f_{0}\right)}}{1-2 r^{2} \cos \left(4 \pi f_{0}\right)+r^{4}} r^{|k|} \cos \left(2 \pi f_{0}|k|-\phi\right)
    $$
    $$
    \phi=\arctan \left[\frac{1-r^{2}}{1+r^{2}} \cot \left(2 \pi f_{0}\right)\right]
    $$
    \begin{gather}
    a[1]=-2 r \cos \left( 2 \pi f_{0}\right)=1.2\\
    a[2]=r^{2}=0.81
    \end{gather}
    这里,
    $
    r \exp \left(\pm j 2 \pi f_{0}\right)
    $是二阶AR的极点。

    The eigenvalues for 4-order autocorrelation matrix are $\begin{bmatrix}
    0.82941 &0.176   &9.33716 &9.77687
    \end{bmatrix}$.

    So choose the $\mu= \dfrac{1/5}{\lambda_{max} \times 2 + \sum_{j=0}^{N} \lambda_j }
    =\dfrac{1/5}{\lambda_{max} \times 2 + 4r_{xx}[0] }=0.005 $.
    \item Plot the curve obtained using (3.41), (3.45), and (3.46), and compare the results.
    \item Compare the measured and theoretical values for the misadjustment.
    \item Calculate the time constans $\tau_{wi}$ and $\tau_{ei}$, and the expected number of iterations to achieve convergence.
    \end{enumerate}
\end{enumerate}

\end{document}